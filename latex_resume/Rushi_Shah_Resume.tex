%%%%%%%%%%%%%%%%%%%%%%%%%%%%%%%%%%%%%%%%%
% "ModernCV" CV and Cover Letter
% LaTeX Template
% Version 1.11 (19/6/14)
%
% This template has been downloaded from:
% http://www.LaTeXTemplates.com
%
% Original author:
% Xavier Danaux (xdanaux@gmail.com)
%
% License:
% CC BY-NC-SA 3.0 (http://creativecommons.org/licenses/by-nc-sa/3.0/)
%
% Important note:
% This template requires the moderncv.cls and .sty files to be in the same 
% directory as this .tex file. These files provide the resume style and themes 
% used for structuring the document.
%
%%%%%%%%%%%%%%%%%%%%%%%%%%%%%%%%%%%%%%%%%

%----------------------------------------------------------------------------------------
%	PACKAGES AND OTHER DOCUMENT CONFIGURATIONS
%----------------------------------------------------------------------------------------

\documentclass[11pt,letterpaper,roman]{moderncv} % Font sizes: 10, 11, or 12; paper sizes: a4paper, letterpaper, a5paper, legalpaper, executivepaper or landscape; font families: sans or roman

\moderncvstyle{casual} % CV theme - options include: 'casual' (default), 'classic'
\moderncvcolor{blue} % CV color - options include: 'blue' (default), 'orange', 'green', 'red', 'purple', 'grey' and 'black'

\usepackage{lipsum} % Used for inserting dummy 'Lorem ipsum' text into the template

\usepackage[scale=.8]{geometry} % Reduce document margins
%\setlength{\hintscolumnwidth}{3cm} % Uncomment to change the width of the dates column
% \setlength{\makecvtitlenamewidth}{10cm} % For the 'classic' style, uncomment to adjust the width of the space allocated to your name

%----------------------------------------------------------------------------------------
%	NAME AND CONTACT INFORMATION SECTION
%----------------------------------------------------------------------------------------

\firstname{Rushi} % Your first name
\familyname{Shah} % Your last name

% All information in this block is optional, comment out any lines you don't need
% \title{Curriculum Vitae}
\address{7609 Leonard Drive}{Falls Church, VA 22043}
% \mobile{(000) 111 1111}
% \phone{(000) 111 1112}
% \fax{(000) 111 1113}
\email{2016rshah@gmail.com}
\homepage{http://www.rshah.org}{Site: rshah.org/} % The first argument is the url for the clickable link, the second argument is the url displayed in the template - this allows special characters to be displayed such as the tilde in this example
\github{2016rshah}

%----------------------------------------------------------------------------------------

\begin{document}

\makecvtitle % Print the CV title
%----------------------------------------------------------------------------------------
%	EDUCATION SECTION
%----------------------------------------------------------------------------------------
\section{Education}

\cventry{Class of 2020}{University of Texas - Austin}{Turing Scholar Honors Program}{}{}{Ranked among US News and World Report's Top 10 Computer Science Programs.}

\cventry{Class of 2016}{Thomas Jefferson High School for Science and Technology (TJHSST)}{}{}{}{Ranked among US News And World Report's Top 5 Public High Schools.}  % Arguments not required can be left empty
%----------------------------------------------------------------------------------------
%	WORK EXPERIENCE SECTION
%----------------------------------------------------------------------------------------

\section{Work Experience}

\cventry{{\color{color1}The MITRE Corporation}}{Computer Science Intern}{Federal Aviation Administration Department}{}{{\color{color1}Python}}{
	\begin{description}
	  \item[Summer 2014] \hfill \\
	  The first year I created and analyzed the performance of various call-sign identification algorithms and conducted natural language processing research on over 12,000 air traffic controller transmissions into the use of prefixes before call signs by air traffic controllers.
	  \item[Summer 2015] \hfill \\
	  The second year I expanded my NLP work to over 25,000 transmissions and analyzed the use of separators between runway identifiers and clearances. I culminated my research by expanding the strict language model used for the Closed Runway Operations Prevention Device (CROPD) to boost the accuracy of the speech recognition engine.
	\end{description}
}

\hfill

\cventry{{\color{color1}Nclud}}{Computer Science Intern}{}{}{{\color{color1}Javascript}}{
	\begin{description}
	  \item[Spring 2015] \hfill \\
	  I was a web-development and web-design intern at Nclud (a DC-based Web Design Firm). While there I assisted on various projects such as the Nclud rebrand, the GreenMachine site, and the extremely important Meteor-Twitter pun app. Most of my work was in JavaScript including the Meteor framework and ThreeJS animations.
	\end{description}
}

\section{Projects}

\cventry{\large{NodeJS}}{Pynt}{}{draw data structures as shapes to get the corresponding Python code. Created at Yale's Hackathon 2014}{\url{https://github.com/Pynt/Pynt}}{}

\cventry{\large{Ruby}}{Github Chart API}{}{embed github contributions calendar into HTML as an image}{\url{https://github.com/2016rshah/githubchart-api}}{}

\cventry{\large{Haskell}}{Heckle}{}{static site compiler that supports LaTeX and Markdown entries}{\newline{} \url{https://github.com/2016rshah/heckle}}{}


%----------------------------------------------------------------------------------------
%	COMPUTER SKILLS SECTION
%----------------------------------------------------------------------------------------

\section{Language Familiarity}

\cvitem{4 years}{\textsc{java}}
\cvitem{3 years}{\textsc{python}, \textsc{javascript}}
\cvitem{2 years}{\textsc{ruby}}
\cvitem{1 year}{\textsc{haskell}}
\cvitem{< 1 year}{\textsc{elixir}}

% %----------------------------------------------------------------------------------------
% %	COVER LETTER
% %----------------------------------------------------------------------------------------

% % To remove the cover letter, comment out this entire block

% \clearpage

% \recipient{Tumblr}{} % Letter recipient
% \date{\today} % Letter date
% \opening{To whom it may concern,} % Opening greeting
% \closing{Respectfully yours,} % Closing phrase

% \makelettertitle % Print letter title

% \makeletterclosing % Print letter signature

%----------------------------------------------------------------------------------------

\end{document}
