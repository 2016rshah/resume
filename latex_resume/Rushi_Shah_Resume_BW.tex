%%%%%%%%%%%%%%%%%%%%%%%%%%%%%%%%%%%%%%%%%
% "ModernCV" CV and Cover Letter
% LaTeX Template
% Version 1.11 (19/6/14)
%
% This template has been downloaded from:
% http://www.LaTeXTemplates.com
%
% Original author:
% Xavier Danaux (xdanaux@gmail.com)
%
% License:
% CC BY-NC-SA 3.0 (http://creativecommons.org/licenses/by-nc-sa/3.0/)
%
% Important note:
% This template requires the moderncv.cls and .sty files to be in the same 
% directory as this .tex file. These files provide the resume style and themes 
% used for structuring the document.
%
%%%%%%%%%%%%%%%%%%%%%%%%%%%%%%%%%%%%%%%%%

%----------------------------------------------------------------------------------------
%	PACKAGES AND OTHER DOCUMENT CONFIGURATIONS
%----------------------------------------------------------------------------------------

\documentclass[11pt,letterpaper,roman]{moderncv} % Font sizes: 10, 11, or 12; paper sizes: a4paper, letterpaper, a5paper, legalpaper, executivepaper or landscape; font families: sans or roman

\moderncvstyle{casual} % CV theme - options include: 'casual' (default), 'classic'
\moderncvcolor{black} % CV color - options include: 'blue' (default), 'orange', 'green', 'red', 'purple', 'grey' and 'black'

\usepackage{lipsum} % Used for inserting dummy 'Lorem ipsum' text into the template

\usepackage[scale=.80]{geometry} % Reduce document margins
\setlength{\hintscolumnwidth}{2.45cm} % Uncomment to change the width of the dates column

% \setlength{\makecvtitlenamewidth}{10cm} % For the 'classic' style, uncomment to adjust the width of the space allocated to your name

\usepackage{framed}
\definecolor{shadecolor}{rgb}{0.85,0.85,0.85}


%----------------------------------------------------------------------------------------
%	NAME AND CONTACT INFORMATION SECTION
%----------------------------------------------------------------------------------------

\firstname{Rushi} % Your first name
\familyname{Shah} % Your last name

% All information in this block is optional, comment out any lines you don't need
% \title{Curriculum Vitae}
% \address{7609 Leonard Drive}{Falls Church, VA 22043}
\email{2016rshah@gmail.com}
\homepage{http://www.rshah.org}{Site: rshah.org/} % The first argument is the url for the clickable link, the second argument is the url displayed in the template - this allows special characters to be displayed such as the tilde in this example
\github{2016rshah}

%----------------------------------------------------------------------------------------

\begin{document}

\makecvtitle % Print the CV title
%----------------------------------------------------------------------------------------
%	EDUCATION SECTION
%----------------------------------------------------------------------------------------
\section{Education}

\cventry{Class of 2020}{University of Texas - Austin}{Turing Scholar Honors Program}{}{}{{Double Majoring in Computer Science and Mathematics. \textbf{GPA: 3.778}}}
%Ranked among US News and World Report's Top 10 Computer Science Programs.
\cventry{Class of 2016}{Thomas Jefferson High School for Science and Technology (TJHSST)}{}{}{}{} 
%Ranked among US News And World Report's Top 5 Public High Schools.
 % Arguments not required can be left empty
%----------------------------------------------------------------------------------------
%	WORK EXPERIENCE SECTION
%----------------------------------------------------------------------------------------

\section{Work Experience}

\cventry{\small{Tokyo, Japan}}{\large{Amazon}}{}{Summer 2018}{}{
	Service to track real-time, location-based purchase trends to tailor recommendations for customers
}

\hfill 

\cventry{\small{New York City}}{\large{Originate}}{}{Summer 2017}{}{
	Distributed computing for data center workload analysis (Scala + Spark + Cassandra)
}

\hfill 


\cventry{\small{Washington DC}}{\large{Nclud}}{}{Spring 2015}{}{
	Full-stack web development (MeteorJS)
}

\hfill

\cventry{\small{N. Virginia}}{\large{The MITRE Corporation}}{Federal Aviation Administration Department}{}{}{
	\begin{description}
	  \item[\small{Summer 2014:}] \hfill \\
	  	%The first year I created and analyzed the performance of various call-sign identification algorithms and conducted natural language processing research on over 12,000 air traffic controller transmissions into the use of 
	  	% prefixes before call signs by air traffic controllers.
	  	- Computational linguistics algorithms for call-sign identification \newline
	  	- Natural language processing research \\
	  	\hspace*{.5cm} - Analyzed emerging patterns in 12,000 air traffic controller transmissions
	  \item[\small{Summer 2015:}] \hfill \\
	  	- Expanded NLP work to over 25,000 transmissions and new emerging patterns\newline
	  	% - Analyzed the use of separators between runway identifiers and clearances\newline
	  	- Edited language model used for the Closed Runway Operations Prevention Device (CROPD)
	\end{description}
}

\section{Other Experience}

\cventry{``UToPiA''}{UT Program Analysis Research Group}{Researcher}{advised by Prof. Isil Dillig}{}{
	Applying program synthesis techniques to database-driven web applications. 
}

\cventry{``ISSS''}{UT Information Systems \& Security Society}{Officer}{}{}{}

\cventry{TX Votes}{TX Votes (non-partisan civic engagement)}{STEM Committee Chairperson}{}{}{
	Organized voter registration drive in CS building (one student registered every three minutes). 
}

\section{Open Source Projects}

\cventry{NodeJS}{Pynt}{}{draw data structures as shapes to get the corresponding Python code}{\newline{} \url{https://github.com/Pynt/Pynt}}{}

\cventry{Haskell}{Heckle}{}{static-site compiler; supports LaTeX/PDF and Markdown/HTML posts}{\newline{} \url{https://github.com/2016rshah/heckle}}{}

% \cventry{Ruby}{Github Chart API}{}{embed github contributions calendar into HTML as an image \hfill }{\url{https://github.com/2016rshah/githubchart-api}}{}


%----------------------------------------------------------------------------------------
%	COMPUTER SKILLS SECTION
%----------------------------------------------------------------------------------------


% \begin{minipage}[t]{0.45\textwidth}
% 	\section{Language Familiarity}
% 		\cvitem{5 years}{\textsc{java}}
% 		\cvitem{4 years}{\textsc{python}, \textsc{javascript}}
% 		\cvitem{3 years}{\textsc{ruby}}
% 		\cvitem{2 years}{\textsc{haskell}}
% 		\cvitem{$\sim 1$ year}{\textsc{scala}, \textsc{c}}
% \end{minipage}
% \begin{minipage}[t]{0.55\textwidth}
	
	\begin{minipage}[t]{0.45\textwidth}
		\section{Selected Coursework}

		\cventry{UT Austin}{Computer Science}{}{}{}{
			\textit{CS 439(H)}: Operating Systems (Honors) \\
			\textit{CS 331(H)}: Algorithms and Complexity (Honors) \\
			\textit{CS 395\ T}: \ Program Verification (Graduate) \\
			\textit{CS\ 389\ L}: \ Automated Logical Reasoning (Graduate)
		}
	\end{minipage}
	\begin{minipage}[t]{0.55\textwidth}
		\subsection{}
		\cventry{}{Math}{}{}{}{
			\textit{M\ \ 341(H)}: Theoretical Linear Algebra (Honors) \\
			\textit{CS\ 311(H)}: Discrete Math (Honors) \\
			\textit{M\ \ 373\ K}: \ Abstract Algebra I
		}
	\end{minipage}

	% \section{Language Familiarity}
	% 	\cventry{}{\textsc{java}, \textsc{python}, \textsc{javascript}, \textsc{ruby}, \textsc{haskell}, \textsc{scala}, \textsc{c}, \textsc{c++}}{}{}{}{}
		
		% \cventry{}{Computer Science}{}{}{}{\textit{CS 439(H)}: Operating Systems (Honors), \textit{CS 331(H)}: Algorithms and Complexity (Honors), \newline{} \textit{CS 395T}: Program Verification (Graduate), \textit{CS\ 389L}: Automated Logical Reasoning (Graduate)}


% 



% %----------------------------------------------------------------------------------------
% %	COVER LETTER
% %----------------------------------------------------------------------------------------

% % To remove the cover letter, comment out this entire block

% \clearpage

% \recipient{Tumblr}{} % Letter recipient
% \date{\today} % Letter date
% \opening{To whom it may concern,} % Opening greeting
% \closing{Respectfully yours,} % Closing phrase

% \makelettertitle % Print letter title

% \makeletterclosing % Print letter signature

%----------------------------------------------------------------------------------------

\end{document}
